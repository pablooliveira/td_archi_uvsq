\documentclass[a4paper,10pt]{exam}

\usepackage[latin9]{inputenc}
\usepackage[cyr]{aeguill}
\usepackage[francais]{babel}
\usepackage{fullpage}
\usepackage{array}

\ifthenelse{\equal{\detokenize{correction}}{\jobname}}
{\printanswers}
{\noprintanswers}

\title{Architecture des ordinateurs - TD 01}
\author{}
\date{Mercredi 2 f�vrier 2011}

\begin{document}
\maketitle
\section{Conversions de base}
\begin{enumerate}
\item Convertissez les nombres d\'ecimaux suivants en binaire :

$255_{10}$, $104_{10}$, $1048576_{10}$, $2010_{10}$

\begin{solution}
 $255_{10} = 1111 1111_{2}$
 
 $104_{10} = 1101000_{2}$
 
 $1048576_{10} = 100000000000000000000_{2}$
 
 $2010 = 11111011010_{2}$
\end{solution}


\item Convertissez les nombres d\'ecimaux suivants en base 5 :

$250_{10}$, $78_{10}$, $33_{10}$, $622_{10}$,

\begin{solution}
$250_{10} = 2000_{5}$
$78_{10} = 303_{5}$
$33_{10} = 113_{5}$
$622_{10} = 4442_{5}$
\end{solution}

\item Convertissez les nombres suivants en d�cimal :

$1234_{5}$, $1234_{7}$, $1234_{9}$

\begin{solution}
$1234_{5} = 194_{10}$
$1234_{7} = 466_{10}$
$1234_{9} = 922_{10}$
\end{solution}


\item Convertissez les nombres binaires suivants en hexad\'ecimal puis en octal:

$110_{2}$, $1011_{2}$, $1100000011011110_{2}$, $11110101100_{2}$

\begin{solution}
 $110_{2} = 6_{16} = 6_{8}$

 $1011_{2} = B_{16} = 13_{8}$

 $1100000011011110_{2} = C0DE_{16} = 140336_{8} $

 $11110101100_{2} = 7AC_{16} = 3654_{8}$

\end{solution}

\item Convertissez les nombres hexad\'ecimaux suivants en d\'ecimal :

$400_{16}$, $A000_{16}$, $FFFF_{16}$, $7FFF_{16}$, $CAFE_{16}$

\begin{solution}
 $400_{16} = 1024_{10}$

 $A000_{16} = 40960_{10}$

 $FFFF_{16} = 65535_{10}$

 $7FFF_{16} = 32767_{10}$

 $CAFE_{16} = 51966_{10}$

\end{solution}

\item Quel est le plus grand entier positif codable sur 9 bits en binaire ? Combien faut-il de chiffres pour l'\'ecrire en octal ? Et en hexad\'ecimal ?

\begin{solution}
$511_{10} (111111111_{2})$. Il faut 3 chiffres pour l'\'ecrire en octal et 3 en hexad\'ecimal.
\end{solution}

\end{enumerate}

\section{Additions et Multiplications en binaire}

\begin{enumerate}
\item Additionnez les entiers positifs suivants directement en binaire, indiquez
  les cas qui produisent un overflow de la repr�sentation 8 bits :

$00101001_2 + 11001010_2$

$10101011_2 + 11001010_2$

$11111111_2 + 11111111_2$

\begin{solution}
 $00101001_2 (41_{10}) + 11001010_2 (202_{10}) = 11110011_2 (243_{10})$.

 $10101011_2 (171_{10}) + 11001010_2 (202_{10}) = 101110101_2 (303_{10})$ le
 r�sultat produit un overflow.

 $11111111_2 (255_{10}) + 11111111_2 (255_{10}) = 111111110_2 (510_{10})$ le
 r�sultat produit un overflow.
\end{solution}

\item Multipliez les entiers positifs suivants, indiquez
  les cas qui produisent un overflow de la repr�sentation 8 bits :

$00001001_2 \times 00001010_2$

$10101011_2 \times 11001010_2$

$11111111_2 \times 11111111_2$

\begin{solution}
 $00001001_2 (9_{10}) \times 00001010_2 (10_{10}) = 01011010_2 (90_2)$.

 $10101011_2 (171_{10}) \times 11001010_2 (202_{10}) = 1000011011101110_2 (34542_{10})$ le
 r�sultat produit un overflow.

 $11111111_2 (255_{10}) \times 11111111_2 (255_{10}) = 1111111000000001_2 (65025_{10})$ le
 r�sultat produit un overflow.
\end{solution}


\item Additionnons un entier $x$ cod� en $n$ bits et un entier $y$ cod� en $m$
  bits, sur combien de bits faut-il coder le r�sultat pour �viter un overflow ?

\begin{solution}

  Supposons sans perte de g�n�ralit� que $n \geq m$.
  Prenons $x_{max} = 2^{n}-1$ et $y_{max} = 2^{m}-1$.
  Or $x+y \leq x_{max} + y_{max} = 2^{n} + 2^{m} - 2 = 2^{n}.(1 + 2^{m-n} - 2^{1-n}) <
  2^{n}.2$, il faut donc $n + 1 = max(n,m) + 1$ bits.
\end{solution}

\item M�me question pour la multiplication avec $n \geq 2$ et $m \geq 2$.

\begin{solution}
  Prenons $x_{max} = 2^{n}-1$ et $y_{max} = 2^{m}-1$.
  Or $x \times y \leq x_{max} \times y_{max} = 2^{n+m} - 2^{n} - 2^{m} + 1 <
  2^{n+m}$, il faut donc au maximum $n+m$ bits.


  Montrons que l'on ne peut pas coder le r�sultat en $n+m-1$ bits,
  soit $2^{n+m} - 2^{n} - 2^{m} + 1 > 2^{n+m-1}$.

  On sait $2^n + 2^m \leq 2^n + 2^n \leq 2^{n+1} \leq 2^{n+m-1}$ car $n \geq m \geq 2$.

  $2^n + 2^m + 2^{n+m-1} \leq 2^{n+m-1} + 2^{n+m-1}$

  $2^n + 2^m + 2^{n+m-1} \leq 2^{n+m}$

  $2^n + 2^m + 2^{n+m-1} -1 < 2^{n+m}$
\end{solution}


\end{enumerate}

\section{Repr\'esentation des binaires cod\'es en d\'ecimal}

\begin{enumerate}
\item Donnez la repr\'esentation en binaire cod\'e en d\'ecimal des nombres suivants :

$89$

$2048$

$1984$

\begin{solution}
$89 = 1000 1001$

$2048 = 0010 0000 0100 1000$

$1984 = 0001 1001 1000 0100$
\end{solution}

\item Donnez la	valeur d\'ecimale des nombres cod\'es en binaire suivants :

$0100 0010$, $0010 0000 0001 0001$, $0101 0001 0010$

\begin{solution}
$0100 0010 = 42$

$0010 0000 0001 0001 = 2011$

$0101 0001 0010 = 512$

\end{solution}

\item Estimez la rapport entre le nombre de bits n\'ecessaires pour coder un nombre en binaire et en binaire cod\'e d\'ecimal.

\begin{solution}
Binaire : Il faut $\lceil log_2(n+1) \rceil$ bits.

BCD : Il faut $4* \lceil log_{10}(n+1) \rceil$ bits.

En applicant la formule $log_{b}(n) = ln(n)/ln(b)$ � ces �quations, on obtient alors un ratio d'environ $4*ln(2)/ln(10)$, soit autour de $1,2$.

\end{solution}

\end{enumerate}

\section{Repr\'esentation en virgule fixe}
Dans la repr�sentation en virgule fixe, un nombre d�cimal s'�crit
avec $n$ chiffres pour la partie enti�re et $q$ chiffres pour la partie
fractionnaire. Voici quelques nombres d�cimaux lorsque $n=2$ et $q=3$:
$12,345$, $05,217$.

\begin{enumerate}
  \item Arrondissez les rationnels suivants � la valeur d�cimale la plus
    proche en virgule fixe ($n=2$ et $q=3$):
    $1/3$
    $15/7$
    \begin{solution}
      $0,333$
      $2,143$
    \end{solution}
  \item Quelle est l'erreur maximale obtenue lors d'un arrondi dans
    la repr�sentation d�cimale en virgule fixe ($n$,$q$) ?
    \begin{solution}
      $0.5 \times 10^{-q}$
    \end{solution}
  \item Quel est le r�sultat de l'op�ration suivante : dans les r�els ? en
    virgule fixe ($n=2$ et $q=3$) ?
    $0,14 \times 0,99$
    \begin{solution}
      $0,1386$

      $0,139$
    \end{solution}
\end{enumerate}

\end{document}
