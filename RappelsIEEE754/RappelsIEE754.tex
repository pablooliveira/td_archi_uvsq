\documentclass[a4paper,10pt]{exam}

\usepackage[latin9]{inputenc}
\usepackage[cyr]{aeguill}
\usepackage[francais]{babel}
\usepackage{fullpage}
\usepackage{amsmath}
\usepackage{array}

\noprintanswers
%\printanswers

\title{IEEE-754 : Aide M�moire}

\author{}
\date{}

\begin{document}
\maketitle

\section{Repr\'esentation des r\'eels}

On consid\`ere la norme IEEE 754 pour repr\'esenter les r\'eels :

\begin{center}
	\begin{tabular}{|>{\centering}p{2cm}|>{\centering}p{3cm}|>{\centering}p{6cm}|}  \hline
  	Signe (S) & Exposant (E) & Pseudomantisse (P)\tabularnewline  \hline
	\multicolumn{1}{>{\centering}p{2cm}}{1 bit} & \multicolumn{1}{>{\centering}p{3cm}}{\textit{e} bits} & \multicolumn{1}{>{\centering}p{6cm}}{\textit{p} bits}\tabularnewline
	\end{tabular}
\end{center}

Le nombre ainsi repr\'esent\'e a pour valeur :

$ (-1)^S * 2^{E - (2^{e-1} - 1)} * (1+\frac{P}{2^p})$

\subsubsection*{Valeurs particuli\`eres}

Les valeurs de l'exposant avec tous les bits \`a 0 ou 1 ($E=00000000$ et $E=11111111$ en 32-bits) sont r\'eserv\'ees pour repr\'esenter des valeurs particuli\`eres, \`a savoir :

\begin{center}
\begin{tabular}{|c|c|c|}
	\hline
	Exposant & Pseudomantisse & Valeur\\	\hline
	Tous les bits \`a 0 & Tous les bits \`a 0 & Repr\'esentation de la valeur z\'ero.\\ \hline
	Tous les bits \`a 0 & Au moins un bit non nul & Repr\'esentation d'une valeur d\'enormalis\'ee. \\ \hline
	Tous les bits \`a 1 & Tous les bits \`a 0 & Repr\'esentation d'une valeur infinie \\ \hline
	Tous les bits \`a 1 & Au moins un bit non nul & Repr\'esentation d'une valeur \textit{NaN} \\ \hline
	\end{tabular}
\end{center}

\begin{itemize}
\item Il y a deux repr\'esentations possibles pour le z\'ero, le z\'ero positif et le z\'ero n\'egatif.

\item Les nombres d\'enormalis\'es sont utilis\'es pour repr\'esenter certaines valeurs sortant de l'intervalle autoris\'e par la repr\'esentation utilis\'ee.

\item Un nombre d\'enormalis\'e a pour valeur :$ (-1)^S * 2^{ 1 - (2^{e-1} - 1)}
  * (\frac{P}{2^p})$. Attention pas de 1 implicite devant la pseudo-mantisse.

\item Une valeur infinie est utilis\'ee pour indiquer par exemple un d\'epassement de capacit\'e. Il y a deux valeurs infinies possibles, positive et n\'egative.

\item Une valeur \textit{NaN} indique toute valeur erronn\'ee, comme par exemple la racine carr\'ee d'un nombre n\'egatif.

\end{itemize}

\subsubsection*{Arrondis}

Si un nombre ne peut pas exactement \^etre repr\'esent\'e en binaire flottant, on utilise une valeur arrondie. Il existe plusieurs modes d'arrondi :
\begin{itemize}
\item L'arrondi au plus proche arrondit \`a la valeur la plus proche. En cas d'\'egalit\'e, on choisit la valeur paire (se terminant par 0 en binaire). C'est le mode d'arrondi par d\'efaut.
\item L'arrondi vers 0 arrondit \`a la valeur la plus proche de 0 (troncature).
\item L'arrondi vers $+\infty$ arrondit \`a la plus proche valeur la plus grande.
\item L'arrondi vers $-\infty$ arrondit \`a la plus proche valeur la plus petite.

\end{itemize}

\subsubsection*{Formats courants}

\begin{center}
\begin{tabular}{|c|c|c|c|c|}
\hline
  & Taille & Signe & Exposant & Pseudomantisse\\ \hline
  Simple pr\'ecision & 32 & 1 & 8 & 23\\ \hline
  Double pr\'ecision & 64 & 1 & 11 & 52\\ \hline
\end{tabular}
\end{center}

Dans la suite, pour simplifier les calculs, les r\'eels seront d\'esormais
repr\'esent\'es sur 16 bits avec la repr\'esentation suivante (ce n'est pas la
m\^eme que la repr\'esentation demi pr\'ecision de la norme IEEE 754) :

\begin{center}
	\begin{tabular}{|>{\centering}p{2cm}|>{\centering}p{3cm}|>{\centering}p{6cm}|}
	\hline
	Signe & Exposant & Mantisse\tabularnewline
	\hline
	\multicolumn{1}{>{\centering}p{2cm}}{1 bit} &
        \multicolumn{1}{>{\centering}p{3cm}}{4 bits} &
        \multicolumn{1}{>{\centering}p{6cm}}{11 bits}\tabularnewline
	\end{tabular}
\end{center}

\section{Codage d'un flottant}
Comment repr�senter $1.32$ en flottant ? $ 1.32 = \frac{132}{100} = \frac{33}{25} $

\begin{center}
\footnotesize
\begin{verbatim}
 100001        | 11001
- 11001        |------        /
-------vv      | 1.01010001111\01...
   100000                     /
  - 11001
   ------vv
      11100
    - 11001
    -------vvvv
         110000
       -  11001
       --------v
          101110
         - 11001
         -------v
           101010
          - 11001
          -------v
            100010
            ....
\end{verbatim}
\end{center}

On dispose de 11 bits pour la pseudo-mantisse, $01010001111$ est l'arrondi au
plus proche.

L'exposant ici est $2^0$ et le signe $0$, donc ce chiffre s'�crira en flottant:
$$ 0\quad 0111 \quad 01010001111 $$

\section{Additions sur les flottants}

L'addition de deux flottants se fait en plusieurs \'etapes. On prendra comme
exemple l'addition des nombres suivants:
\begin{align}
0\quad 0111\quad 10000000000 =& + 1.1 \times 2^{7-(2^{4-1}-1)} = + 1.1_{2} \times
2^{0} = 1.5\\
0\quad 0110\quad 00000000000 =& + 1.0 \times 2^{6-(2^{4-1}-1)} = + 1.0_{2} \times
2^{-1} = 0.5
\end{align}

\begin{enumerate}
\item On ram\`ene les deux nombres au m\^eme exposant, le plus grand des deux.
  On d\'ecale donc les bits de la mantisse du nombre ayant le plus petit
  exposant d'autant de bits vers la droite que la diff\'erence entre les
  exposants, sans oublier le 1 avant la virgule.

  Dans l'exemple on veut augmenter de 1 l'exposant du deuxi�me terme.  La
  mantisse compl\`ete du second nombre passe donc de $1.00000000000$ \`a
  $0.10000000000$ (on supprime le z\'ero le plus \`a droite pour rester sur 11
  bits).

\item On ajoute ensuite les deux mantisses, en tenant compte du signe (attention
  ici les mantisses ne sont pas exprim\'ees en compl\'ement \`a 2)
  Dans l'exemple les mantisses sont de m�me signe on peut donc les ajouter directement:
  $1.10000000000 + 0.10000000000 = 10.00000000000$

\item On renormalise ensuite le nombre obtenu.
  Dans l'exemple, le r\'esultat a donc pour exposant $0$ et pour mantisse $10.000000000$ :
  on renormalise la mantisse, ce qui augmente l'exposant de 1. On a donc le
  r\'esultat final :
  $$0\quad 1000\quad 00000000000 = + 1.0_{2} \times 2^{1} = 2$$
\end{enumerate}

\section{Multiplication sur les flottants}

Supposons que l'on multiplie les nombres suivants:
\begin{align}
0\quad 1000\quad 01000000000 &= + 1.01_{2} \times 2^{1} = 10.1_{2} = 2.5\\
0\quad 1000\quad 10000000000 &= + 1.1_{2} \times 2^{1} = 11_{2} = 3
\end{align}

La multiplication se fait en plusieurs �tapes:
\begin{enumerate}

\item D\'etermination du bit de signe : il vaut 1 si les bits de signe des deux r\'eels sont diff\'erents, et 0 s'ils sont identiques.
Ici, les bits de signe sont identiques : le bit de signe du r\'esultat sera donc $0$


\item L'exposant est la somme des exposants respectifs des r\'eels.

Les exposants des deux r\'eels sont tous les deux 1 : l'exposant du r\'esultat
sera donc $1+1=2$, soit $1001_{2}$ en codage par exc\`es.  Cela peut aussi se
faire en ajoutant les exposants tels qu'ils apparaissent dans la
repr\'esentation binaire et en y soustrayant l'exc\'edent.
  $$1000_{2} + 1000_{2} - 111_{2} = 8 + 8 - 7 = 9 = 1001_{2}$$

\item On multiplie les mantisses entre elles. Ceci peut s'effectuer d'une fa\c{c}on similaire \`a une multiplication en base 10.

 La multiplication des mantisses $1.01000000000$ et $1.1000000000$ peut s'\'ecrire :
$$1.01 \times 1.1 = 1.01  \times (1.0 + 0.1) = 1.01 + 0.101 = 1.111$$ soit $1.11100000000$ pour la mantisse du r\'esultat.
On en d\'eduit l'\'ecriture du r\'esultat : $0\quad 1001\quad 11100000000$, ce qui correspond bien \`a l'\'ecriture de $7.5$
\end{enumerate}


\end{document}
