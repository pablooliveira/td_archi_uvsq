\documentclass[a4paper,10pt]{exam}

\usepackage[utf8]{inputenc}
\usepackage[cyr]{aeguill}
\usepackage[francais]{babel}
\usepackage{fullpage}
\usepackage{amsmath}
\usepackage{array}

\ifthenelse{\equal{\detokenize{correction}}{\jobname}}
{\printanswers}
{\noprintanswers}

\title{Architecture des ordinateurs - TD 04}

\author{}
\date{}

\begin{document}
\maketitle

\section{Codage excédent 127}

La repr\'esentation sur 8 bits en excédent 127 consiste à
encoder la valeur $x + 127$ en binaire.
Par exemple pour encoder $-27$ on prendra $1100100_2 = 100_{10}$.

\begin{enumerate}
\item Quelle est la valeur minimale et maximale codable en excédent 127
  sur 8 bits ?
  \begin{solution}
    -127 et 128
  \end{solution}
  \item Donner la représentation en excédent 127 des nombres suivants:
    10, 23 et -40
    \begin{solution}
      $10 = 10001001$

      $23 = 10010110$

      $-40 = 01010111$
    \end{solution}
  \item Calculer $1001_{exc} + 11000011_{exc}$ en excédent 127:
    \begin{enumerate}
      \item en convertissant chaque terme en décimal, puis en convertissant
        le résultat en excédent 127.
      \item proposer une méthode qui ne nécessite pas le passage en décimal.
    \end{enumerate}
    \begin{solution}
      $-118 + 68 = -50 = 1001101$

      $1001 + 11000011 - 1111111 = 11001100 - 01111111 = 01001101 $
    \end{solution}
\end{enumerate}

\section{Repr\'esentation des r\'eels}
\begin{enumerate}

\item Donnez la plus grande et la plus petite valeur strictement positives repr\'esentables en simple pr\'ecision (normalis\'e). M\^eme question avec un nombre d\'enormalis\'e.

\begin{solution}
Minimum = $0 00000001 0000000000000000000000$, soit $2^{-126}$.

En d\'enormalis\'e : $0 00000000 000000000000000000000001 = 2^{-148}$

Maximum = $0 11111110 1111111111111111111111$, soit $2^{127}*(2-2^{-23}) \simeq 2^{128}$

\end{solution}

Pour simplifier, dans la suite on utilisera la représentation suivante:

\begin{center}
	\begin{tabular}{|>{\centering}p{2cm}|>{\centering}p{3cm}|>{\centering}p{6cm}|}
	\hline
	Signe & Exposant & Mantisse\tabularnewline
	\hline
	\multicolumn{1}{>{\centering}p{2cm}}{1 bit} &
        \multicolumn{1}{>{\centering}p{3cm}}{4 bits} &
        \multicolumn{1}{>{\centering}p{6cm}}{11 bits}\tabularnewline
	\end{tabular}
\end{center}


\item Exprimez  les nombres d\'ecimaux suivants (on utilisera l'arrondi par d\'efaut si n\'ecessaire) :

$1.5$

$-0.125$

$153.75$

$-0.2$

\begin{solution}
\begin{align*}
1.5 =& 0\quad 0111\quad 10000000000\\
-0.125 =& 1\quad 0100\quad 00000000000\\
153.75 =& 0\quad 1110\quad 00110011100\\
-0.2 =& 1\quad 0100\quad 10011001100
\end{align*}
\end{solution}

\item Convertissez en d\'ecimal les nombres flottants suivants  :

$0\quad 1110\quad 00000110010$

$1\quad 0001\quad 00101100000$

$0\quad 0000\quad 00100100000$

$0\quad 1111\quad 10000000001$

\begin{solution}

\begin{enumerate}
  \item $131.125 = 1.0244140625 \times 2^{7}$
  \item $0.018310546875 = 1.171875 \times 2^{-6}$
  \item dénormalisé: $0.00439453125 (0.28125 \times 2^{-6})$
  \item NaN
\end{enumerate}

\end{solution}

\item Quelle est la repr\'esentation en simple pr\'ecision (sur 32 bits) des nombres suivants, exprim\'es en double pr\'ecision (sur 64 bits) :

$4004000000000000_{16}$

$37E8000000000000_{16}$

$C800000000000000_{16}$

\begin{solution}
$4004000000000000 (= 2.5) = 40200000$ en simple

$37E8000000000001 (= (1.5+\epsilon)*2^{-129}) = 00180000$ en simple

$C800000000000000_{16} (=-2^{129})$ overflow, \'eventuellement $FF800000$ ($-\infty$)

\end{solution}

\end{enumerate}

\pagebreak
\section{Sommes de flottants}

\begin{enumerate}

\item Effectuez les calculs suivants en utilisant la repr\'esentation des r\'eels :

$0\quad  0101\quad  10111000011 + 0\quad  0011\quad  00011110101$

$1\quad  1101\quad  10010001000 + 0\quad  0110\quad  10000000000$

\begin{solution}
\begin{verbatim}
1)
On décale la mantisse du deuxième nombre de 2:
  1.10111000011
+ 0.01000111101
---------------
 10.00000000000 * 2^-2

On renormalise 0 0110 00000000000 = 0.5

2)
On décale la mantisse du deuxième nombre de 7:
0.00000011000 - 1.10010001000 = -(1.10010001000 - 0.00000011000)
                              = - 1.10001110000
car
  1.10010001000
- 0.00000011000
---------------
  1.10001110000

Le résultat (déjà normalisé): 1 1101 10001110000 = -99.5
\end{verbatim}
\end{solution}
\end{enumerate}
\section{Multiplication de flottants}
\begin{enumerate}
\item Effectuez les multiplications suivantes:
$0\quad 1001\quad 00110000000 \times 0\quad  0111\quad  10000000000$

$1\quad 1100\quad 00000000000 \times 0\quad  0110\quad  01000000000$
\begin{solution}
\begin{verbatim}
1)
  Le signe du résultat est positif (0)
  L'exposant est la somme des exposants : 2 + 0 = 2
        1.0011
      * 1.1000
      --------
             0
            0
           0
      10011
     10011
     ---------
    1.11001000

  Le résultat est donc 0 1001 11001000000 = 7.125

2)
  Le signe du résultat est négatif (1)
  L'exposant est la somme des exposants : 5 - 1 = 4
  On multiplie les mantisses:
     1.01 * 1.0 = 1.01

  Le résultat est donc 0 1011 01000000000 = 20
\end{verbatim}

\end{solution}

\item Repr\'esentez les nombres $3$ et $\frac{1}{3}$ en utilisant les arrondis
  au plus pr\`es, vers 0, vers $+\infty$ et vers $-\infty$. Donnez ensuite les valeurs d\'ecimales correspondantes.

\begin{solution}
\begin{itemize}
\item L'arrondi au plus proche arrondit \`a la valeur la plus proche. En cas d'\'egalit\'e, on choisit la valeur paire (se terminant par 0 en binaire). C'est le mode d'arrondi par d\'efaut.
\item L'arrondi vers 0 arrondit \`a la valeur la plus proche de 0 (troncature).
\item L'arrondi vers $+\infty$ arrondit \`a la plus proche valeur la plus grande.
\item L'arrondi vers $-\infty$ arrondit \`a la plus proche valeur la plus petite.

\end{itemize}
$3 = 0 1000 10000000000$ dans tous les cas
$\frac{1}{3} = 0 0101 01010101010$ en arrondi vers 0, et vers $-\infty$ (en
d\'ecimal : $0.333252$), $0 0101 01010101011$  au plus pr\`es et vers $+\infty$ (en d\'ecimal : $0.333374$)
\end{solution}

\item Donnez le r\'esultat de la multiplication de ces deux nombres.

\begin{solution}
On ajoute les exposants, on multiplie les mantisses, et on normalise.

Au plus pr\`es et vers $+\infty$, les mantisses donnent $10.00000000000$ et l'exposant deviendrait $-1$, on a donc $0 0111 00000000000$ (soit $1$)

En arrondi vers 0 et $-\infty$, les mantisses donnent $1.11111111110$ et l'exposant deviendrait $-1$, donc $0 0110 11111111111$ (soit $0.999756$)

\end{solution}

\end{enumerate}



\end{document}
