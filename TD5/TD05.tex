\documentclass[a4paper,10pt]{exam}

\usepackage[utf8]{inputenc}
\usepackage[cyr]{aeguill}
\usepackage[francais]{babel}
\usepackage{fullpage}
\usepackage{amsmath}
\usepackage{array}
\usepackage{tikz}
\input kvmacros
\usetikzlibrary{arrows,shapes,trees,patterns,fit,backgrounds,%
decorations.pathreplacing,chains,calc,decorations.pathmorphing,matrix}

\ifthenelse{\equal{\detokenize{correction}}{\jobname}}
{\printanswers}
{\noprintanswers}

\title{Architecture des ordinateurs - TD 05}

\author{}
\date{}

\begin{document}
\maketitle

\section{Algèbre de Boole: simplifications}
\begin{enumerate}
  \item Démontrer les propriétés suivantes:
    \begin{align}
      X(\overline{X} + Y) = XY \\
      (X+Y)(X+Z) = X + YZ \\
      XY + X\overline{Y} = X \\
      (X+Y)(X+\overline{Y}) = X \\
      XY + \overline{Y} = X + \overline{Y}
    \end{align}
    \begin{solution}
      \begin{align*}
        X(\overline{X} + Y) = X\overline{X} + XY = 0 + XY \\
        (X+Y)(X+Z) = XX + XZ + YX + YZ = X(1 + Y + Z) + YZ = X.1 + YZ \\
        XY + X\overline{Y} = X(Y + \overline{Y}) = X.1 \\
        (X+Y)(X+\overline{Y}) = X + Y\overline{Y} (\textrm{car (2)}) = X + 0\\
        X + \overline{Y} = (X + \overline{Y})(Y + \overline{Y}) = \overline{Y} + XY
      \end{align*}
    \end{solution}
  \item Simplifier les expressions suivantes:
    \begin{align}
      AB\overline{C} + \overline{(AB\overline{C})}\\
      (AB + C\overline{D})(AB + \overline{D}E)\\
      A + \overline{B}C + \overline{D}(A+\overline{B}C)\\
      A\overline{B}(C + D) + \overline{(C + D)}\\
      (\overline{(EF)} + AB + \overline{C}\overline{D})(EF)\\
      (AB + C) + (D + EF)\overline{(AB + C)}
  \end{align}
  \begin{solution}
    \begin{align*}
      AB\overline{C} + \overline{(AB\overline{C})} = \Delta + \overline{\Delta}
      = 1 \\
      (AB + C\overline{D})(AB + \overline{D}E) = AB + C\overline{D}E
      \textrm{(car (2))}\\
      A + \overline{B}C + \overline{D}(A+\overline{B}C) = \Delta +
      \overline{D}\Delta = \Delta = (A + \overline{B}C)\\
      A\overline{B}(C + D) + \overline{(C + D)} = A\overline{B} + \overline{(C +
        D)} \textrm{car (5)}\\
      (\overline{(EF)} + AB + \overline{C}\overline{D})(EF) = (EF)(AB +
      \overline{C}\overline{D})\\
      (AB + C) + (D + EF)\overline{(AB + C)} = (AB + C) + (D + EF)
  \end{align*}
  \end{solution}

\item Donner le tableau de Karnaugh correspondant à la table de vérité ci-dessous.
Simplifier F.
\begin{center}
\begin{tabular}{cc|c}
P&Q&F\\
\hline
0&0&1\\
0&1&1\\
1&0&0\\
1&1&1\\
\end{tabular}
\end{center}

\begin{solution}
  \begin{center}\begin{tikzpicture}
    \matrix (karnaugh) [matrix of math nodes] {
      1 & 1 \\
      0 & 1 \\
    } ;

    \foreach \i/\bits in {1/0,2/1} {
      \node [above = 2mm of karnaugh-1-\i] {$\bits$} ;
      \node [left  = 2mm of karnaugh-\i-1] {$\bits$} ;
    }

    \node [left  = .6cm of karnaugh.west]  {$P$} ;
    \node [above = .5cm of karnaugh.north] {$Q$} ;

    \begin{pgfonlayer}{background}
      \begin{scope}[opacity=.5]
        \draw [thick]
              (karnaugh-1-1.north west) rectangle (karnaugh-1-2.south east)
              (karnaugh-1-2.north west) rectangle (karnaugh-2-2.south east);
      \end{scope}
    \end{pgfonlayer}
  \end{tikzpicture}\end{center}

  $$
  Q+\overline{P}
  $$
\end{solution}

\item Donner le tableau de Karnaugh correspondant à la table de vérité ci-dessous.
Simplifier F.
\begin{center}
\begin{tabular}{ccc|c}
A&B&C&F\\
\hline
0&0&0&0\\
0&0&1&1\\
0&1&0&1\\
0&1&1&0\\
1&0&0&1\\
1&0&1&1\\
1&1&0&0\\
1&1&1&1\\
\end{tabular}
\end{center}

\begin{solution}
  \begin{center}\begin{tikzpicture}
    \matrix (karnaugh) [matrix of math nodes] {
      0 & 1 \\
      1 & 1 \\
      0 & 1 \\
      1 & 0 \\
    } ;

    \foreach \i/\bits in {1/0,2/1} {
      \node [above = 2mm of karnaugh-1-\i] {$\bits$} ;
    }

    \foreach \i/\bits in {1/00,2/01,3/11,4/10} {
      \node [left  = 2mm of karnaugh-\i-1] {$\bits$} ;
    }

    \node [left  = .6cm of karnaugh.west]  {$BC$} ;
    \node [above = .5cm of karnaugh.north] {$A$} ;

    \begin{pgfonlayer}{background}
      \begin{scope}[opacity=.5]
        \draw [thick]
        (karnaugh-1-2.north west) rectangle (karnaugh-3-2.south east)
        (karnaugh-2-1.north west) rectangle (karnaugh-2-2.south east)
        (karnaugh-4-1.north west) rectangle (karnaugh-4-1.south east);
      \end{scope}
    \end{pgfonlayer}
  \end{tikzpicture}\end{center}

  $$
  A\overline{B} + AC + \overline{B}C + \overline{A}B\overline{C}
  $$
\end{solution}

\end{enumerate}

\pagebreak
\section{L'assassinat de la Duchesse}
Un crime exécrable a été commis au manoir: la Duchesse Alice a été assassinée.
L'inspecteur de police mêne son enquête.  Un des occupants ment.
L'inspecteur note les déclarations suivantes:
\begin{enumerate}
  \item Bob: je n'ai pas tué la Duchesse.
  \item Celine: Dimitri et Eric ont tué la Duchesse.
  \item Dimitri: Céline et Eric ont tué la Duchesse.
  \item Eric: Céline n'a pas tué la Duchesse.
\end{enumerate}
Qui est coupable ?
Construisez une table de verité pour aider l'inspecteur.

\begin{solution}
  Soit X la proposition: \og X a tué la duchesse\fg{}.
  Soit x la proposition: \og X dit la verité\fg{}.

  \begin{tabular}{ccccccccc}
    BCDE&b&c&d&e&un seul menteur\\
    0000&1&0&0&1&0\\
    0001&1&0&0&1&0\\
    0010&1&0&0&1&0\\
    0011&1&1&0&1&1\\
    0100&1&0&0&0&0\\
    0101&1&0&1&0&0\\
    0110&1&0&0&0&0\\
    0111&1&1&1&0&1\\
    1000&0&0&0&1&0\\
    1001&0&0&0&1&0\\
    1010&0&0&0&1&0\\
    1011&0&1&0&1&0\\
    1100&0&0&0&0&0\\
    1101&0&0&1&0&0\\
    1110&0&0&0&0&0\\
    1111&0&1&1&0&0\\
  \end{tabular}

  Il y a deux solutions:
  \begin{itemize}
    \item C,D,E ont tué la duchesse.
    \item D,E ont tué la duchesse.
  \end{itemize}

  Suspens: Céline est-elle coupable ?
\end{solution}

\section{Problème du coffre fort}

Une banque s'équipe d'un nouveau coffre-fort.
Le coffre-fort ne peut-être ouvert que par:
\begin{itemize}
  \item le directeur et le sous-directeur ensemble.
  \item le directeur, le comptable et le fondé de pouvoir de la banque.
  \item le sous-directeur, le comptable et le fondé de pouvoir.
\end{itemize}

Combien faut-il installer de serrures au minimum sur ce coffre ?
Comment répartir les clefs de ces serrures parmis les personnes
ci-dessus ?

\begin{enumerate}
\item
  Soit $D$ une variable booléenne valant $1$ lorsque le directeur est
  présent pour l'ouverture du coffre et $0$ dans le cas contraire.  On définira
  de la même manière $S,C,F$ pour les trois autres responsables de la banque.
  Soit une fonction $F(D,S,C,F) \rightarrow \{0,1\}$. $F$ vaut 1 si et seulement
  si le coffre doit être ouvert. Par exemple $F(0,0,1,1) = 0$: le comptable
  et le fondé de pouvoir ne peuvent ouvrir le coffre à eux seuls.

 Ecrire l'équation booléene ou la table de vérité caractérisant la fonction $F$.
\begin{solution}
  $F(D,S,C,F) = D.S + D.C.F + S.C.F$
\end{solution}

\item Ecrire la fonction $F$ sous la forme $F = \Sigma(n,m,\dots)$ où $n,m,\dots$
  est la représentation décimale des minterms.
\begin{solution}
  $F(D,S,C,F) = \Sigma(7,11,12,13,14,15)$
\end{solution}

\item Écrire le tableau de Karnaugh de $\overline{F}$.
\item Trouver grâce au tableau de Karnaugh, la forme normale disjonctive
  minimale de $\overline{F}$.
\item En utilisant les lois de Morgan, trouvez la forme normale conjonctive
  minimale de $F$.
\item Combien de serrures faut-il installer ? Comment répartir les clefs parmi
  le personnel ?
\begin{solution}

  \begin{center}\begin{tikzpicture}
    \matrix (karnaugh) [matrix of math nodes] {
      1 & 1 & 1 & 1 \\
      1 & 1 & 0 & 1 \\
      0 & 0 & 0 & 0 \\
      1 & 1 & 0 & 1 \\
    } ;

    \foreach \i/\bits in {1/00,2/01,3/11,4/10} {
      \node [left  = 2mm of karnaugh-\i-1] {$\bits$} ;
      \node [above = 2mm of karnaugh-1-\i] {$\bits$} ;
    }

    \node [left  = .6cm of karnaugh.west]  (DS) {$DS$} ;
    \node [above = .5cm of karnaugh.north] (CF) {$CF$} ;

    \begin{pgfonlayer}{background}
      \begin{scope}[opacity=.5]
        \draw [thick]
              (karnaugh-1-1.north west) rectangle (karnaugh-2-2.south east);
        \draw [thick]
              (karnaugh-1-1.north west) rectangle (karnaugh-1-4.south east);
        \fill [blue!40]
              (karnaugh-1-1.north west) rectangle (karnaugh-1-1.south east)
              (karnaugh-4-4.north west) rectangle (karnaugh-4-4.south east)
              (karnaugh-1-4.north west) rectangle (karnaugh-1-4.south east)
              (karnaugh-4-1.north west) rectangle (karnaugh-4-1.south east);
        \fill [red!20]
              (karnaugh-1-4.north west) rectangle (karnaugh-2-4.south east)
              (karnaugh-1-1.north west) rectangle (karnaugh-2-1.south east);
        \fill [green!20]
              (karnaugh-1-1.north west) rectangle (karnaugh-1-2.south east)
              (karnaugh-4-1.north west) rectangle (karnaugh-4-2.south east);
      \end{scope}
    \end{pgfonlayer}
  \end{tikzpicture}\end{center}

$$
  \overline{F} = \overline{D}.\overline{C} + \overline{D}.\overline{S} +
\overline{F}.\overline{S}
  + \overline{S}.\overline{C} + \overline{D}.\overline{F}
$$

$$
  F = (D+C).(D+S).(F+S).(S+C).(D+F)
$$

Il faut donc 5 serrures:
\begin{itemize}
  \item D reçoit les clefs 1,2,5.
  \item S reçoit les clefs 2,3,4
  \item C reçoit les clefs 1,4
  \item F reçoit les clefs 3,5
\end{itemize}
\end{solution}

\end{enumerate}


% \section{Afficheur 7 segments}
% On souhaite réaliser un circuit pour afficher les chiffres de 0 à 9 sur un
% afficheur 7 segments (voir figure ci-dessous).  Un afficheur 7 segments, est
% composés de 7 leds commandées par 7 signaux distincts $A_1 \dots A_7$.  Le
% chiffre à afficher est codé en BCD, chaque bit correspond à un signal $b_3$ à
% $b_0$.

% Nous souhaitons réaliser le circuit permettant de commander le signal $A_3$
% correspondant à l'allumage du segment inférieur.

% \vspace{1cm}
% \begin{minipage}{0.4\textwidth}
% \begin{center}
% Codage BCD

% \begin{tabular}{c|c}
%   déc. & $b_3b_2b_1b_0$ \\
%   \hline
%   0 & 0000\\
%   1 & 0001\\
%   2 & 0010\\
%   3 & 0011\\
%   4 & 0100\\
%   5 & 0101\\
%   6 & 0110\\
%   7 & 0111\\
%   8 & 1000\\
%   9 & 1001
% \end{tabular}
% \end{center}
% \end{minipage}
% \begin{minipage}{0.4\textwidth}
% \begin{center}
% Afficheur 7 segments

% \begin{tikzpicture}
%   \draw (0.1,0) -- node[above] {$A_1$} (0.9,0);
%   \draw (0.1,-1) -- node[above] {$A_2$} (0.9,-1);
%   \draw (0.1,-2) -- node[below] {$A_3$} (0.9,-2);

%   \draw (0,-0.1) -- node[left] {$A_4$} (0,-0.9);
%   \draw (1,-0.1) -- node[right] {$A_5$} (1,-0.9);

%   \draw (0,-1.1) -- node[left] {$A_6$} (0,-1.9);
%   \draw (1,-1.1) -- node[right] {$A_7$} (1,-1.9);
% \end{tikzpicture}
% \end{center}
% \end{minipage}

% \begin{enumerate}
%   \item Écrire la table de vérité du signal $A_3$.
%     Lorsque la valeur de $A_3$ est indeterminée écrire X (par exemple pour B=13).
%     \begin{solution}
%       \begin{tabular}{cccc|c}
%         $b_3$& $b_2$& $b_1$& $b_0$& $A_3$\\
%         \hline
%         0&0&0&0&1\\ %0
%         0&0&0&1&0\\ %1
%         0&0&1&0&1\\ %2
%         0&0&1&1&1\\ %3
%         0&1&0&0&0\\ %4
%         0&1&0&1&1\\ %5
%         0&1&1&0&1\\ %6
%         0&1&1&1&0\\ %7
%         1&0&0&0&1\\ %8
%         1&0&0&1&0\\ %9
%         1&0&1&0&X\\ %10
%         1&0&1&1&X\\ %11
%         1&1&0&0&X\\ %12
%         1&1&0&1&X\\ %13
%         1&1&1&0&X\\ %14
%         1&1&1&1&X\\ %15
%       \end{tabular}
%     \end{solution}
%   \item Écrire le tableau de Karnaugh correspondant à la table de vérité de
%     $A_3$.
%     \begin{solution}
%       \begin{tabular}{c|c|c|c|c|}

%       \end{tabular}
%     \end{solution}
% \end{enumerate}


\end{document}
