\documentclass[a4paper,10pt]{exam}

\usepackage[latin9]{inputenc}
\usepackage[cyr]{aeguill}
\usepackage[francais]{babel}
\usepackage{fullpage}
\usepackage{array}
\usepackage{verbatim}
\usepackage{amsmath}

\ifthenelse{\equal{\detokenize{correction}}{\jobname}}
{\printanswers}
{\noprintanswers}

\title{Architecture des ordinateurs - TD 02}

\author{}
\date{}

\begin{document}
\maketitle

\section{Repr\'esentation d'entiers sign\'es}
\begin{enumerate}
\item Donnez sur 8 bits les repr\'esentations signe plus valeur absolue, compl\'ement \`a 1 et compl\'ement \`a 2 des nombres d\'ecimaux suivants :

$36_{10}$, $-123_{10}$, $-45_{10}$

\begin{solution}
$36_{10} = 00100100_{2}$ dans les 3 \'ecritures.

$-123_{10} = 11111011_{2}$ en signe+abs, $10000100_{2}$ en compl\'ement \`a 1 et $10000101_{2}$ en compl\'ement \`a 2

$-45_{10} = 10101101_{2}$ en signe+abs, $11010010_{2}$ en compl\'ement \`a 1 et $11010011_{2}$ en compl\'ement \`a 2

\end{solution}

\item Quelle est la valeur en d\'ecimal des nombres binaires dont la repr\'esentation en compl\'ement \`a 2 sur 8 bits est :

$11111111_{2}$, $01001000_{2}$, $10001111_{2}$

Quelles seraient ces valeurs si ces nombres binaires repr�sentaient un compl\'ement \`a 1 ? Un signe plus valeur absolue ?
\begin{solution}
$11111111_{2} = -1_{10}$ si cplment 2, $0_{10}$ si cplment 1 et $-127_{10}$ si sign+valeur abs.

$01001000_{2} = 72_{10}$ dans les 3 modes

$10001111_{2} = -113_{10}$ si cplment 2, $-112_{10}$ si cplment 1 et $-15_{10}$ si sign+valeur abs.

\end{solution}


\item Consid\'erons les repr\'esentations en compl\'ement \`a 2 de x, y et z sur 8 bits :

$10010101_{2}$, $00010010_{2}$, $11101011_{2}$

Sans calculer la valeur de x, y et z en d\'ecimal, donnez leur repr\'esentation en compl\'ement \`a 2 sur 16 bits.

Quel est le nombre minimal de bits pouvant repr\'esenter x en compl\'ement \`a 2 ? M\^eme question pour y et z.

\begin{solution}
$x=1111111110010101$, $y=0000000000010010$, $z=1111111111101011$

Pour x, 8 bits. Pour y et z, 6 bits.

\end{solution}

\item D\'eterminez les valeurs maximales et minimales repr\'esentables en compl\'ement \`a 2 sur n bits.

\begin{solution}
$-2^{n-1} \leq x \leq 2^{n-1}-1$

\end{solution}

\item Comment identifierez-vous un nombre pair sous sa repr\'esentation binaire en signe + valeur absolue ? En compl\'ement \`a 1 ? En compl\'ement \`a deux ?

\begin{solution}
En signe + val abs, il finit par 0, ainsi qu'en compl\'ement \`a 2. En compl\'ement \`a 1, le bit de signe et le dernier signe doivent \^etre identiques.
\end{solution}

\end{enumerate}

\section{Additions / soustractions d'entiers sign\'es}
\begin{enumerate}
\item Transformez la soustraction de deux entiers sign\'es en une addition dans la repr\'esentation en compl\'ement \`a 2 puis calculez $120_{10}-45_{10}$.

\begin{solution}
$120_{10}-45_{10} = 120_{10} + -45_{10} = 01111000_{2} + 11010011_{2} = 01001011_{2}$
\end{solution}


\item Sans passer par leur repr\'esentation d\'ecimale, effectuer les op\'erations suivantes sur des nombres de 8 bits en compl\'ement \`a deux :

$00100100_{2} + 01000000_{2}$

$00011001_{2} + 11001110_{2}$

$11110110_{2} + 10100110_{2}$

$00101101_{2} + 01101111_{2}$

$10000001_{2} + 11000000_{2}$

\begin{solution}
Pour les overflow : c'est si les deux derniers bits de retenue sont diff\'erents.

En compl\'ement \`a 2 :
$00100100_{2} + 01000000_{2} = 01100100_{2}$

$00011001_{2} + 11001110_{2} = 11100111_{2}$

$11110110_{2} + 10100110_{2} = 110011100_{2}$

$00101101_{2} + 01101111_{2} = 010011100_{2}$ (overflow)

$10000001_{2} + 11000000_{2} = 101000001_{2}$ (overflow)

\end{solution}

\item En compl\'ement \`a 2, quel est le nombre de bits minimum n\'ecessaires pour effectuer l'op\'eration $125_{10} + 5_{10}$ ?

\begin{solution}
9 bits (sinon overflow)
\end{solution}

\item Le compl�ment � 2 d'un nombre positif $x = x_{n-1}2^{n-1} + \dots + x_{0}2^{0}$ cod� sur $n$ bits peut-�tre d�fini de deux mani�res:
  \begin{itemize}
    \item $CA2(x) = 2^n-x$
    \item $CA2(x) = \overline{x} + 1$ (le compl�ment � 1 plus un).
  \end{itemize}
  Montrez que les deux d�finitions sont �quivalentes.

  \begin{solution}

Soit $x = x_{n-1}2^{n-1} + \dots + x_{0}2^{0}$, posons
\begin{align*}
2^n - x =& 2^n -  (x_{n-1}2^{n-1} + \dots + x_{0}2^{0})\\
2^n - x =& (2^{n-1} + \dots + 2^{0} +1)  - (x_{n-1}2^{n-1}+\dots +x_{0}2^{0})\\
2^n - x =& (1-x_{n-1})2^{n-1} + \dots + (1-x_{0})2^{0} +1\\
2^n - x =& (\overline{x_{n-1}}2^{n-1} + \dots + \overline{x_{0}}2^{0} +1\\
2^n - x =& \overline{x} + 1\\
\end{align*}

  \end{solution}

  \end{enumerate}

\section{Un peu de programmation : cardinal d'un ensemble repr�sent� avec un
  champs de bits}

Soit un ensemble $E \in \mathcal{P}([|0;31|])$ par exemple $E' = \{4,5,8,15\}$.
Une repr�sentation tr�s compacte de cet ensemble est bas�e sur l'utilisation
d'un champs de bit (\emph{bit set} en anglais). Dans cette repr�sentation on
encode $E$ sous la forme d'un mot de $32$ bits $M=(m_{31}m_{30}\dots m_{0})_2$.
On utilise la convention convention suivante : $m_n = 1$ si et seulement si $n
\in E$.  Par exemple, pour repr�senter l'ensemble $E'$ on utilisera le mot $M'=
00000000000000001000000100110000_2$.

Nous souhaitons �crire une fonction \texttt{char get\_size(uint32\_t M);} qui
calcule le cardinal de l'ensemble $E$. Cela revient � compter le nombre de bits
� 1.

\begin{enumerate}
\item �crire la fonction \texttt{get\_size} en utilisant une boucle et
  l'op�rateur $>>$.  Dans le pire cas vous effectuerez $32$ it�rations.
\begin{solution}
\begin{verbatim}
char get_size(uint32_t M) {
    char count = 0;
    while(M) {
	    count += M & 1; // incr�menter si le dernier bit est � 1
	    M = M >> 1; // d�caler m � droite
    }
    return count;
}
\end{verbatim}
\end{solution}

\item Consid�rez le mot $M'$ ci-dessus. Calculez $M'' = M' \& (M'-1)$.
Calculez $M''' = M'' \& (M''-1)$. Que remarquez vous ?
\begin{solution}
$M'' = 00000000000000001000000100110000_2 \& (00000000000000001000000100101111_2)
     = 00000000000000001000000100100000_2$

$M''' = 00000000000000001000000100000000_2$

� chaque application de la formule on mets � z�ro le bit le plus � droite.
\end{solution}

\item Utilisez le r�sultat de la question pr�c�dente pour r��crire la
fonction \texttt{get\_size} en utilisant une boucle qui effectue au
maximum $card(E)$ it�rations.

\begin{solution}
\begin{verbatim}
char get_size_faster(uint32_t M) {
    char count = 0;
    while(M) {
	    M = M & (M-1); // on annule le bit � 1 le plus � droite.
	    count += 1;
    }
    return count;
}
\end{verbatim}
\end{solution}

\item Question ouverte : voyez vous des algorithmes plus efficaces pour calculer
  $card(E)$ ?
\end{enumerate}
\begin{solution}
  Si on pr�calcule dans un tableau le nombre de bits � 1 dans des mots de 8 bits.
  Une possiblit� bien plus rapide consiste � d�couper l'entier en 4 mots de 8
  bits et utiliser la table pour obtenir le r�sultat.

  Bien entendu, on pourrait faire la m�me chose directement avec une table de 32
  bits, mais sa taille serait alors prohibitive.
\end{solution}

\end{document}
